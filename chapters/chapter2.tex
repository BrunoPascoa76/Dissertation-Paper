\chapter{Estado da Arte}
\label{chap:estadodaarte}

\begin{introduction}
Este capítulo descreve o estado da arte atual relativamente à monitorização do estado emocional do utilizador, bem como arquiteturas e metodologias de processamento contínuo de dados, identificando o trabalho já realizado nestas áreas, bem como as lacunas identificadas.
\end{introduction}

\section{Sensores e Monitorização do Estado Emocional}

\section{Lacuna Identificada}
Apesar dos progressos identificados acima na monitorização do estado emocional do utilizador através de sensores, eu pude verificar que a maioria das soluções referidas concentram-se na recolha pontual de dados. Até à data, não encontrei exemplos concretos de soluções que permitam a recolha de dados de vários sensores de forma contínua e escalável.

\section{Arquiteturas de Processamento Contínuo de Dados (Streaming)}

O processamento contínuo de dados, ou streaming, consiste na ingestão, análise e armazenamento de dados à medida que são gerados (ao contrário de processamento em batch, no qual os dados são processados apenas após o final de uma sessão ou período pré-definido), permitindo monitorizar e analisar estes dados em tempo real. Nas subsecções seguintes serão apresentados alguns dos componentes utilizados no streaming de dados que serão relevantes para esta dissertação.

\subsection{Message Brokers}

Message Brokers são sistemas que permitem a receção de mensagens em filas divididas por tópicos, permitindo assim uma comunicação por eventos entre produtores (que enviam os dados) e consumidores (que os enviam e recebem). Entre as tecnologias mais usadas destacam-se o \ac{MQTT} (utilizado em brokers como \href{https://mosquitto.org/documentation/}{Eclipse Mosquitto}, por exemplo) e o Apache Kafka.

O \ac{MQTT} é um protocolo de mensagens publish-subscribe conhecido pela sua simplicidade, sendo muitas vezes utilizado em sistemas \ac{IoT} devido ao baixo consumo de energia e largura de banda, permitindo uma comunicação eficiente mesmo em sistemas embutidos com recursos limitados~\cite{MQTT_IOT}.

Por outro lado, o Kafka é um message broker que utiliza um esquema distribuído em clusters de modo a proporcionar alta performance, escalabilidade e tolerância a falhas~\cite{kafka_aws,kafka_datapeaker}. O Kafka Connect, componente integrado no próprio Kafka, permite também importar e exportar dados entre o Kafka e outras fontes de dados (incluindo MQTT), permitindo a transmissão de dados de forma fluida e eficiente~\cite{kafka_connect_docs}.

\subsection{Time-Series Databases}

Bases de Dados Time-Series são base de dados NO-SQL conhecidas por serem otimizadas para indexarem dados num certo momento no tempo, sendo por isso a escolha mais apropriada para registar valores medidos por sensor.
Neste caso, a base de dados mais relevante para este contexto é o InfluxDB, visto ter sido desenhada de raiz para lidar com elevados volumes de dados temporais~\cite{time_series_comparison}, sendo por isso o ideal para armazenamento de dados de sensores.

\subsection{Processamento de Dados em Paralelo}