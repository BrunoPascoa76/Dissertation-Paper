\chapter{Introdução}
\label{chap:introduction}

\begin{introduction}
Este capítulo apresenta uma introdução ao tema da dissertação ao contextualizar a sua motivação e descrever os objetivos e metodologia.
\end{introduction}

\section{Contexto e Motivação}

Atualmente, dispositivos pessoais como computadores, smartphones ou smartwatches possuem diversos sensores que permitem recolher informações úteis para inferir os hábitos e o estado psicológico do utilizador. Estes sensores registam não só dados biométricos, como a frequência cardíaca, mas também informações sobre como o utilizador interage com estes dispositivos, como padrões do uso teclado ou de movimento do rato, permitindo monitorização do estado psicológico do utilizador em tempo real.
\section{Objetivos}
Os principais objetivos deste trabalho consistem em desenvolver um sistema modular capaz de recolher dados de vários sensores, processá-los em streaming e disponibilizar essa informação para visualização ou para estimar automaticamente o estado psicológico do utilizador.

Apesar de trabalhos anteriores estabelecerem relações entre um (ou mais) sensor e o estado psicológico do utilizador, estes sistemas limitavam-se à recolha de dados em cada sessão, carecendo da escalabilidade e de mecanismo de gestão de sensores que seriam esperados num protótipo funcional capaz de integrar múltiplas sessões e sensores de forma contínua.

Neste contexto, o presente trabalho propõe o desenvolvimento de um sistema modular escalável baseado em streaming capaz de recolher dados de vários sensores de forma contínua, gerir automaticamente estes dispostivos e integrar facilmente novos sensores ou módulos.

\section{Metodologia}

Neste trabalho irei adotar uma metodologia baseada na monitorização do estado psicológico do utilizador (particularmente o nível de atenção) durante o uso do computador. Neste cenário, dados serão recolhidos através de sensores do próprio computador (teclado e câmara), bem como sensores externos (\ac{HRV} via smartwatch). Estes dados serão processados numa solução à base de stream compatível com cloud.

O sistema será projetado de forma a ser escalável e extensível, com mecanismo de gestão e monitorização automática de dispositivos e mantendo a possibilidade de integração de novos componentes sem que isto afete o restante sistema.

\section{Estrutura do Documento} %TODO: atualizar quando houver mais capítulos
Este documento está organizado em seis capítulos. O Capítulo~\ref{chap:introduction} apresenta uma introdução, contextualizando a sua motivação, descrevendo os objetivos pretendidos e a metodologia adotada para os atingir.
O Capítulo~\ref{chap:estadodaarte} apresenta o estado da arte atual, bem como trabalhos relacionados a esta dissertação.
O Capítulo~\ref{chap:metodos} especifica a arquitetura e requisitos funcionais e não funcionais do sistema.
O Capítulo~\ref{chap:planotrabalho} detalha o trabalho preliminar realizado no primeiro semestro e os seus resultados, bem como uma proposta de plano de trabalho para o segundo semestre.
O Capítulo~\ref{chap:resultados} descreve os resultados obtidos, bem como o processo de validação da solução final.
Finalmente, o Capítulo~\ref{chap:conclusions} sumariza as contribuições, discute as limitações e sugere direções para trabalhos futuros.



